\documentclass[8pt]{article}
\usepackage{enumerate}
\usepackage{indentfirst}
\usepackage[margin=0.75in]{geometry}
\usepackage[pdfborder={0 0 0}]{hyperref}
\usepackage[normalem]{ulem}

\usepackage{listings}
\lstset{
  basicstyle=\ttfamily,
  mathescape
  }

\renewcommand{\thefootnote}{\fnsymbol{footnote}}

\begin{document}
\title{PLINK 2 File Format Specification Draft}
\maketitle
\begin{quote}\small
  The master version of this document can be found at
  \url{https://github.com/chrchang/plink-ng/tree/master/pgen_spec}. \\
\end{quote}

\newpage
\tableofcontents

\newpage
\section{Introduction}
PLINK 1's binary genotype file format (described at
\url{https://www.cog-genomics.org/plink/1.9/formats#bed}) is simple, compact,
and supports direct computation on the packed data representation.  Thanks to
these properties, it continues to be widely used more than a decade after it
was designed.

However, it can only represent unphased biallelic hard-called genotypes.  This
limitation makes it suboptimal for an increasing fraction of genome-wide
association studies, which tend to benefit from inclusion of imputed genotype
dosages and more sophisticated handling of multiallelic variants.  In addition,
the inability to represent phase information is problematic for investigation
of compound heterozygosity, imputation-related data management, and several
other workflows.

The most widely-used binary genotype file format which addresses these
limitations is BCF (\url{https://github.com/samtools/hts-specs}).
Unfortunately, while it is more machine-friendly than VCF, BCF does not support
direct computation on packed data, so it's impossible to write software for it
which can consistently match or exceed PLINK 1.9's efficiency.

Therefore, PLINK 2 introduces the PGEN binary genotype file format, a
backward-compatible extension which can represent phased, multiallelic, and
dosage data in a manner that provides much better support for ``compressive
genomics''\footnote{\url{https://www.nature.com/articles/nbt.2241}}.  It also
incorporates
SNPack\footnote{\url{http://sysbiobig.dei.unipd.it/?q=Software\#SNPack}}-style
genotype compression, frequently reducing file sizes by 80+\% with negligible
encoding and decoding cost (and supporting some direct computation on the
compressed representations).  The result is, of course, not as simple as the
PLINK 1 format, but the open-source \texttt{pgenlib\_internal} library (used by
PLINK 2) provides one way to ignore the additional complexity.

PLINK 2 also introduces backward-compatible extensions to PLINK 1's sample and
variant text formats.  PSAM (extension of .fam) supports storage of categorical
(e.g. 1000 Genomes's ``Population'' field) and other phenotype / covariate
data.  PVAR (extension of .bim) supports storage of all header and
variant-specific information in a typical VCF, and is defined such that
``sites-only VCF'' files are valid PVARs requiring no conversion by PLINK 2 at
all.

\newpage
\section{PGEN Format Specification}

PGEN is a binary format capable of representing mixed-phase, multiallelic,
mixed-hardcall/dosage/missing genotype data.

A PLINK 1 variant-major .bed file is also a valid PGEN file.  Yes, this is
because it's grandfathered in as a special case.  But it's a special case that
doesn't take much additional code to handle, since the usual basic
representation of hard-called biallelic genotype data is almost identical.

PGEN(+PVAR) is designed to interoperate with, not replace, VCF/BCF.  PGEN
cannot represent auxiliary per-genotype-call data such as read depths or
quality scores; or biallelic genotype probability triplets; or triploid
genotypes.  It is just designed to represent the subset of the VCF format which
is relevant to PLINK's domains of function.  Considerable effort has been spent
on optimizing PLINK 2's VCF $\leftrightarrow$ PGEN conversion functions.

\subsection{Overall file organization}

A PGEN file is composed of a mandatory header, followed by a sequence of
variant records.  The header contains enough information to enable random
access to the variant records.

A variant record's main data track can be ``LD-compressed'' (LD = linkage
disequilibrium), referring to the most recent non-LD-compressed variant record
and only storing genotype-category differences from it.  This is the only type
of inter-record dependency, so it doesn't take that much memory to process a
PGEN file sequentially: other than the record type and size information in the
header, only the genotypes from the latest non-LD-compressed variant might need
to be kept for possible future reference.

A future version of this specification may add an optional footer, containing
phase-set and possibly other information.

It is not always possible to write a PGEN file in a single sequential pass,
since the header usually contains variant record lengths and compression
methods which aren't known until variant-record-writing time.  The
\texttt{pgenlib\_internal} implementation uses a two-pass approach, computing
only the size of the header on the first write pass, and writing the actual
contents at the end.  Three fixed-width storage modes are defined (covering
basic unphased biallelic genotypes, unphased dosages, and phased dosages) which
don't have this limitation, and are especially straightforward to read and
write; but they don't benefit from PGEN's low-overhead genotype compression.  A
future version of this specification may add a way to store most header
information in a separate file, so that sequential reading, sequential writing,
and genotype compression are simultaneously possible (at the cost of more
annoying file management).

\subsection{Header}

\subsubsection{Magic number}

All PGEN files start with the two magic bytes \texttt{0x6c 0x1b}.

\subsubsection{Storage mode}

The third byte indicates the overall storage mode.  The following modes are
currently defined:

\begin{itemize}
  \itemsep0em
\item \texttt{0x01} is PLINK 1 variant-major .bed.  In this case, there are no
  more header bytes; the number of samples is not stored in the file, and must
  be known by some other means when initializing a PGEN reader.
\item \texttt{0x02} is the simplest PLINK 2 fixed-width format.  All variant
  records are of type \texttt{0}.  (Variant record types are discussed in
  section \ref{sec:vr}.)
\item \texttt{0x03} is the fixed-width unphased-dosage format.  All variant
  records are of type \texttt{0x40}.
\item \texttt{0x04} is the fixed-width phased-dosage format.  All variant
  records are of type \texttt{0xc0}.
\item \texttt{0x10} is the standard PLINK 2 format, with variable-width variant
  records.
\end{itemize}

Mode values \texttt{0x05..0x0f} and \texttt{0x11..0x7f} are reserved for use by
future versions of this specification, and mode value 0 is off-limits since it
corresponds to a PLINK 1 sample-major .bed file (and is detected and
automatically converted to regular PGEN by PLINK 2).  \texttt{0x80..0xff} can
be safely used by developers for their own purposes.

\subsubsection{Dataset dimensions, header body formatting}

If the storage mode isn't \texttt{0x01}, the next eight bytes are the number of
variants (as a little-endian \texttt{uint32}; everything else in this
specification is also little-endian), followed by the number of samples as a
\texttt{uint32} (call this $N$).  This is followed by a byte describing how the
rest of the header is formatted.

\begin{itemize}
  \itemsep0em
\item Bits 0-3 of the twelfth byte indicate how variant-record types and
  lengths are stored.  Interpreting them as a single number in 0..15, these are
  the meanings:
  \begin{itemize}
  \item \texttt{0}: 4 bits per record type, 1 byte per record length.
  \item \texttt{1}: 4 bits per record type, 2 bytes per record length.
  \item \texttt{2}: 4 bits per record type, 3 bytes per record length.
  \item \texttt{3}: 4 bits per record type, 4 bytes per record length.
  \item \texttt{4}: 8 bits per record type, 1 byte per record length.
  \item \texttt{5}: 8 bits per record type, 2 bytes per record length.
  \item \texttt{6}: 8 bits per record type, 3 bytes per record length.
  \item \texttt{7}: 8 bits per record type, 4 bytes per record length.
  \item \sout{\texttt{8}: 0 bits per record type, 2 bits per record length ($\ell
    -\lceil N/4\rceil$ is stored when the record length is $\ell$ byte(s)).
    Record type is inferred as 0 when record length is minimal, and 8 (i.e.
    multiallelic-variant hard-call) when record length has 1-3 extra bytes.
    This is designed for single-sample filesets sharing a single PVAR.}
  \item \sout{\texttt{9}: 0 bits per record type, 4 bits per record length ($\ell
    -\lceil N/4\rceil$ stored when record length is $\ell$).  Record type is
    inferred as 0 when record length is minimal, and 8 when record-length has
    1-15 extra bytes.}
  \end{itemize}
  Values 8-15 are reserved for future use.

  With storage modes \texttt{0x02..0x04}, these bits are always zeroed out,
  since it is unnecessary to store any additional variant-record type or length
  information.
  \item Bits 4-5 indicate how many bytes are used to store each allele count.
    Zero indicates that no allele-count information is stored: either there are
    no multiallelic variants, or their allele counts must be known by some
    other means (e.g. loading the accompanying PVAR) before PGEN reading.
  \item Bits 6-7 indicate how ``provisional reference'' flag information is
    stored.  \texttt{0} indicates that this information isn't in the PGEN (and
    should be present in the accompanying PVAR).  \texttt{1} indicates that
    no reference alleles are provisional (all are trusted since they were e.g.
    imported from a VCF).  \texttt{2} indicates that \textit{all reference
    alleles are provisional} (none are trusted, usually because the PGEN was
    created from a PLINK 1 fileset which isn't designed to track REF/ALT
    alleles).  \texttt{3} indicates that some but not all reference alleles are
    provisional, and this is tracked by bitarrays in the header.
\end{itemize}

Storage modes defined in the future may use more than one byte to describe
header formatting.

\subsubsection{Variant block offsets}

For the variable-width storage mode (\texttt{0x10}), the PGEN file is divided
into ``variant blocks'' of $2^{16}$ variants.  (The last block may contain
fewer variants.) Thus, if there are $M$ variants, there are
$B := \lceil M/2^{16}\rceil$ variant blocks.  The next $8B$ bytes of the header
store the (0-based) offsets of each variant block within the file, as
\texttt{uint64}s.

This subsection is not present for the fixed-width modes.

\subsubsection{Main header body}

For the variable-width storage mode (\texttt{0x10}), the header body is also
divided into blocks corresponding to $2^{16}$ variants each.  Each block
contains a packed array of $2^{16}$ variant record types, followed by a packed
array of variant record lengths, possibly followed by an array of allele
counts, and then possibly followed by a provisional-REF-allele flag bitarray.
All of these arrays end at byte boundaries; when there are unused bits at the
end of any of these arrays, they must be set to zero.  Both properties are also
true for all packed arrays of 1/2/4-bit elements mentioned later in this
specification, except when explicitly stated otherwise.

For the fixed width storage modes, only the provisional-REF-allele flag
bitarray might be present.  (It can be interpreted as being divided into blocks
corresponding to $2^{16}$, or not; doesn't make a difference.)

\subsubsection{Example}

Suppose we're generating a standard PGEN from a PLINK 1 fileset with 1092
samples and 39728178 variants.

\begin{itemize}
\item Bytes 0-2 are \texttt{0x6c 0x1b 0x10}, the magic number followed by a
  byte indicating the usual compression-supporting storage mode.  (In the rest
  of this specification, phrases like ``byte $a$-$b$'', ``byte $c$'', ``record
  \#$d$'', and ``block \#$f$'' use 0-based indexing.  Only
  ``first''/``second''/``third''/``$n$th'' uses 1-based indexing.)
\item Bytes 3-6 are \texttt{0x32 0x34 0x5e 0x02}, the little-endian binary
  representation of 39728178.
\item Bytes 7-10 are \texttt{0x44 0x04 0x00 0x00}, the little-endian binary
  representation of 1092.
\item Byte 11 is probably \texttt{0x81}, indicating 4 bits per record type, 2
  bytes per record length, no allele counts (they're unnecessary since all
  variants are biallelic), and no trusted REF alleles (since we're generating
  this from a PLINK 1 fileset where REF/ALT may not be tracked).  It would be
  \texttt{0x41} if the PLINK 1 fileset always had A2=REF and we made PLINK 2
  aware of this with the \texttt{--real-ref-alleles} flag.
\item Bytes 12-19 store the starting position of variant record \#0 in the
  file; bytes 20-27 store the starting position of variant record \#65536;
  \ldots ; bytes 4860-4867 store the starting position of variant record
  \#39714816.
\item Bytes 4868-37635 store the first $2^{16}$ variant record types.  The low
  4 bits of byte 4868 are the type of record \#0, the high 4 bits of byte 4868
  are the type of record \#1, the low 4 bits of byte 4869 are the type of
  record \#2, etc.
\item Bytes 37636-168707 store the first $2^{16}$ variant record byte lengths.
\item Bytes 168708-201475 store the types of variant records 65536-131071, and
  bytes 201476-332547 store the lengths; \ldots ; bytes 99291908-99298588 store
  the types of variant records 39714816-39728177, and bytes 99298589-99325312
  store the lengths.
\item Variant record \#0 starts at byte 99325313.  Thus, bytes 12-19 are
  \texttt{0x81 0x95 0xeb 0x05 0x00 0x00 0x00 0x00}.
\end{itemize}

Note that the variant-block starting positions stored in bytes 12-4867 make it
possible to extract an arbitrary variant record from the middle without reading
most of the 99.3 MB header.  For example, to extract record \#31000000, it's
useful to read bytes 3796-3803 to get the starting position of record
\#30998528; it's necessary to read byte 77501924 to get the variant record
type; and it's necessary to read bytes 77533956-77536902 to get the lengths of
\textit{every} record from \#30998528 to \#31000000, since these lengths are
needed to compute the start and end positions of record \#31000000 given the
start position of record \#30998528.  This last part is annoying, but it would
be far more annoying if we had to add 31 million numbers together instead of
less than 1500 of them.

\subsection{Variant records}
\label{sec:vr}

Each variant record starts with the main data track, which is sufficient to
describe unphased biallelic hard-calls.  The following 10 auxiliary data tracks
may also appear (in the listed order), depending on the variant record type:

\begin{enumerate}
\item Multiallelic hard-calls.
\item Hardcall-phase information.
\item Biallelic dosage existence.
\item Biallelic dosage values.
\item Multiallelic dosage existence.
\item Multiallelic dosage values.
\item Biallelic phased-dosage existence.
\item Biallelic phased-dosage values.
\item Multiallelic phased-dosage existence.
\item Multiallelic phased-dosage values.
\end{enumerate}

If the file is PLINK 1 variant-major .bed (storage mode \texttt{0x01}), all
variant records are of PLINK 1's type, which doesn't have an associated numeric
code, and no auxiliary data tracks may be present.

Otherwise, bits 0-2 of the variant record type describe how the main data track
is stored, bit 3 indicates whether multiallelic hard-calls are present, bit 4
indicates whether hardcall-phase information is present, bits 5 and 6 describe
whether and how unphased dosage data is stored, and bit 7 indicates whether
explicit phased-dosages are present.

Variant records are currently limited to 4284736160 (slightly under $2^{32}$)
bytes.

\subsubsection{Difflists}

Before describing the main data track in more detail, it is useful to define
what a PGEN ``difflist'' is.  This construct appears several times in the
subsections to follow.

A difflist represents either a possibly empty increasing sequence of sample
IDs, or a possibly empty sequence of (sample ID, 2-bit genotype category value)
pairs.  (Sample ID values in PGEN files are always integers in [0, $N-1$],
where $N$ is the number of samples.)  As the name suggests, it is designed to
represent a sparse list of differences from something else.  It does so in a
manner that is compact, and supports fast checking of whether a specific sample
ID is in the list.

All difflists start with a base-128 varint (see
\url{https://developers.google.com/protocol-buffers/docs/encoding#varints}),
indicating how many elements are in the list; call this value $L$.  If $L=0$,
that's the end of the list.

Otherwise, the elements are divided into groups of 64 (the last group may be
smaller); call the number of groups $G:=\lceil L/64\rceil $.  The remaining
components of the difflist are, in order:

\begin{enumerate}
\item Sample IDs \#0, \#64, \#128, \#192, \ldots (i.e. the first ID in each
  group), stored as \texttt{uint8}s iff $N\leq 2^8$, \texttt{uint16}s iff
  $2^8<N\leq 2^{16}$, \texttt{uint24}s iff $2^{16}<N\leq 2^{24}$, and
  \texttt{uint32}s otherwise.
\item $G-1$ bytes, where byte $j$ in this component is 63 less than the byte
  size of group \#$j$ in the final component.  (63 is subtracted because the
  minimum possible raw value is 63, and the maximum possible value is larger
  than 255 but not larger than 255+63.)  To reduce overhead for short lists,
  the byte size of the last group is not stored in this manner.
\item If 2-bit genotype category values are also stored in the difflist, the
  fourth component is a packed array of them, occupying $\lfloor L/4\rfloor $
  bytes.
\item The final component is a sequence of $L-G$ varints encoding ([sample ID
  \#$k$] - [sample ID \#$k-1$]) for all positive $k<L$ which aren't multiples
  of 64.
\end{enumerate}

As an example, suppose $N=488377$, $L=79$, 2-bit genotype category values are
also stored in the difflist, and the sample IDs in the list are 5000, 10000,
15000, \ldots , 395000.  Then:

\begin{itemize}
\item The first byte of the difflist is \texttt{0x4f}, 79 encoded as a
  varint (which is just plain 79, since $79<128$).
\item The next six bytes store sample IDs \#0 and \#64 in the list, using
  \texttt{uint24} encoding (i.e. \texttt{uint32} with the last byte omitted).
  \texttt{0x88 0x13 0x00 0x88 0xf5 0x04}.
\item The next byte is the final-component byte size of the first group (we'll
  see that this is 126) minus 63.  \texttt{0x3f}.
\item The next 20 bytes store the packed genotype category values associated
  with the sample IDs in the difflist.  More precisely, the bottom two bits of
  the first byte are for sample \#5000, bits 2-3 of the first byte are for
  sample \#10000, bits 4-5 of the first byte are for sample \#15000, bits 6-7
  of the first byte are for sample \#20000, bits 0-1 of the second byte are for
  sample \#25000, \ldots .
\item The remainder of the difflist stores ([sample ID \#1] - [sample ID \#0]),
  ([sample ID \#2] - [sample ID \#1]), \ldots , ([sample ID \#63] - [sample ID
  \#62]), ([sample ID \#65] - [sample ID \#64]), ([sample ID \#66] - [sample ID
  \#65]), \ldots , ([sample ID \#78] - [sample ID \#77]) using varint encoding.
  There are 77 values (note that there is no entry for $k=64$), all equal to
  5000 in this example, so these 154 bytes are \texttt{0x88 0x27 0x88 0x27 0x88
  0x27 \ldots}, since 5000's varint encoding is \texttt{0x88 0x27}.
\end{itemize}

\subsubsection{Main data track}

The main data track distinguishes between homozygous-REF, heterozygous REF-ALT,
double-ALT, and missing hard-calls.  These 4 categories are sufficient to
describe all unphased biallelic hard-call possibilities, while still making
sense for multiallelic variants.

If the variant record is of PLINK 1's type, the four categories are encoded in
a packed array of 2-bit values as \texttt{0} = double-ALT, \texttt{1} =
missing, \texttt{2} = heterozygous REF-ALT, and \texttt{3} = homozygous-REF.
Otherwise, the 2-bit encoding is \texttt{0} = homozygous-REF, \texttt{1} =
heterozygous REF-ALT, \texttt{2} = double-ALT, and \texttt{3} = missing, and
the bottom three bits of the variant record type indicates what compression is
used on the resulting packed array.

\begin{itemize}
\item \texttt{0}: no compression.
\item \texttt{1}: ``1-bit'' representation.  This starts with a byte indicating
  what the two most common categories are (value \texttt{1}: categories 0 and
  1; \texttt{2}: 0 and 2; \texttt{3}: 0 and 3; \texttt{5}: 1 and 2; \texttt{6}:
  1 and 3; \texttt{9}: 2 and 3); followed by a bitarray describing which
  samples are in the higher-numbered category; followed by a difflist with all
  (sample ID, genotype category value) pairs for the two less common
  categories.
\item \texttt{2}: LD-compressed.  A difflist with all (sample ID, genotype
  category value) pairs for samples in different categories than they were in
  in the previous non-LD-compressed variant.  The first variant of a variant
  block (i.e. its index is congruent to 0 mod $2^{16}$) cannot be
  LD-compressed.
\item \texttt{3}: LD-compressed, inverted.  A difflist with all (sample ID,
  genotype value) pairs for samples in different categories than they would be
  in the previous non-LD-compressed variant after inversion (categories 0 and 2
  swapped).  This addresses spots where the reference genome is ``wrong'' for
  the population of interest.
\item \texttt{4}: Difflist with all (sample ID, genotype category value) pairs
  for samples outside category 0.
\item \texttt{5}: Reserved for future use.  (When all samples appear to be in
  category 1, that usually implies a systematic variant calling problem.)
\item \texttt{6}: Difflist with all (sample ID, genotype category value) pairs
  for samples outside category 2.
\item \texttt{7}: Difflist with all (sample ID, genotype category value) pairs
  for samples outside category 3.
\end{itemize}

\subsubsection{Multiallelic hard-calls}

When bit 3 of the variant record type is set, multiallelic hard-call(s) are
present involving an ALT allele after the first.  In this case, all category 1
hard-calls are initially assumed to be REF-ALT1 (ALT1 = first ALT allele), and
category 2 hard-calls are initially assumed to be homozygous-ALT1; then the
``patch sets'' described below are applied to fill in hard-calls involving
ALT2/ALT3/etc.

The low 4 bits of the first byte of auxiliary data track \#1 indicates how the
category 1 patch set is formatted.

\begin{itemize}
\item \texttt{0}: Patch set starts with a bitarray with $J$ bits, where $J$ is
  the number of samples in genotype category 1, and bit $j\in [0, J)$ is set
  iff the ($j+1$)th smallest sample ID among those in genotype category 1 isn't
  a REF-ALT1 genotype.  Let $K$ be the number of set bits; then the bitarray is
  followed by a packed array of $K$ fixed-width values, where the width depends
  on the total number of ALT alleles, and each value is 2 less than the ALT
  allele index in the corresponding genotype:
  \begin{itemize}
  \item 2 ALT alleles: width ZERO.  All set bits in the initial bitarray
    correspond to REF-ALT2.
  \item 3 ALT alleles: width 1 bit.  I.e. set bits in the packed array
    correspond to REF-ALT3, clear bits correspond to REF-ALT2.
  \item 4-5 ALT alleles: width 2 bits.
  \item 6-17 ALT alleles: width 4 bits.
  \item 18-257 ALT alleles: width 8 bits.
  \item 258-65537 ALT alleles: width 16 bits.
  \item 65538-16777215 ALT alleles: width 24 bits.
  \end{itemize}
\item \texttt{1}: Same as format 0, except that the initial bitarray is
  replaced with a difflist containing the sample IDs with category 1,
  not-REF-ALT1 genotypes.
\item \texttt{2..14}: These values are reserved for future use.
\item \texttt{15}: Empty patch set; everything in category 1 is REF-ALT1.
\end{itemize}

The high 4 bits of the first byte indicates how the category 2 patch set is
formatted.  The actual patch set appears after the end of the category 1 patch
set.

\begin{itemize}
\item \texttt{0}: Patch set starts with a bitarray with $J$ bits, where $J$ is
  the number of samples in genotype category 2, and bit $j\in [0, J)$ is set
  iff the ($j+1$)th smallest sample ID among those in genotype category 2 isn't
  a homozygous-ALT1 genotype.  Let $K$ be the number of set bits.  Then, if
  there are exactly 2 ALT alleles, this is followed by a bitarray with $K$ bits
  where set bits correspond to homozygous-ALT2 genotypes, and clear bits
  correspond to heterozygous ALT1-ALT2.  Otherwise, the initial bitarray is
  followed by a packed array of $K$ pairs of fixed-width values, where the
  width depends on the total number of ALT alleles, each value is 1 less than
  the ALT allele index in the corresponding genotype, and the first value in
  each pair is never larger than the second:
  \begin{itemize}
  \item 3-4 ALT alleles: width 2 bits (so each genotype requires 4 bits).  For
    example, an ALT1-ALT3 heterozygous call would be represented by the 4-bit
    value 8: the high 2 bits are equal to 2, representing ALT3, and the low 2
    bits are equal to 0, representing ALT1.
  \item 5-16 ALT alleles: width 4 bits (each genotype is 8 bits).
  \item 17-256 ALT alleles: width 8 bits (each genotype is 16 bits).
  \item 257-65536 ALT alleles: width 16 bits (each genotype is 32 bits).
  \item 65537-16777215 ALT alleles: width 24 bits (each genotype is 48 bits).
  \end{itemize}
\item \texttt{1}: Same as format 0, except that the initial bitarray is
  replaced with a difflist containing the sample IDs with category 2,
  not-homozygous-ALT1 genotypes.
\item \texttt{2..14}: These values are reserved for future use.
\item \texttt{15}: Empty patch set; everything in category 2 is
  homozygous-ALT1.
\end{itemize}

\subsubsection{Phased heterozygous hard-calls}

When bit 4 of the variant record type is set, phased heterozygous hard-calls
are present.

In this case, the first \textit{bit} of auxiliary data track \#2 indicates
whether an explicit ``phasepresent'' bitarray is stored.  Let $H$ be the number
of heterozygous hard-calls; when this initial bit is set, the next $H$ bits
(including the high 7 bits of the first byte) are 1-bit values where 0 = no
hardcall-phase for that heterozygous call, and 1 = the call is phased.  When
the initial bit is clear, all heterozygous hard-calls are phased.

This is followed by a ``phaseinfo'' bitarray, which begins at the usual byte
boundary when ``phasepresent'' is stored, but begins at bit 1 of the first
auxiliary data track \#2 byte instead when phasepresent is omitted.  It has $P$
bits, where $P$ is the number of phased heterozygous hard-calls; clear bits
correspond to ``unswapped'' alleles (first allele index is less than the
second, like ``0$|$1'' in the VCF GT field), and set bits correspond to
``swapped'' alleles.

Note that the PGEN format does not distinguish between GT values of ``0/0'' and
``0$|$0'', since the distinction is meaningless.  (Phase sets are meaningful,
but they can't consistently be represented by switching between ``0/0'' and
``0$|$0'' anyway.)

\subsubsection{Dosage existence and values}

When bits 5 and 6 of the variant record type are zero, no dosage data is
present; auxiliary data tracks \#3-4 (and \#5-10, for that matter) don't exist.

Otherwise, auxiliary data track \#4 contains dosages, represented as an array
of \texttt{uint16}s.  In this array, a value in [0, $2^{15}$] corresponds to a
sum-of-ALT-allele-dosages equal to that value multiplied by $2^{-15}\cdot $
[maximum possible dosage]; e.g. 1638 corresponds to a diploid ALT allele dosage
of roughly 0.05, or a haploid dosage of roughly 0.025.  (Note that PGEN files
don't distinguish between diploid and haploid calls; it is assumed that ploidy
is known from other context.)

\textbf{Dosages are not allowed to be inconsistent with the corresponding
hard-calls.}  More precisely, given a (possibly phased) dosage, the (possibly
phased) hard-call must be either missing, or have Manhattan distance $\leq 0.5$
from the dosage (typesetted definition of Manhattan dosage distance TBD).

Note that some multiallelic dosages don't have any hard-calls within Manhattan
distance 0.5; and maximum-likelihood variant calling can also be expected to
produce a few biallelic-variant hard-calls which violate this consistency
criterion.  If an analyst is unwilling to part with these marginal hard-calls,
they are expected to keep them in a separate PGEN from the corresponding
dosages.

It usually isn't necessary to store a dosage equal to exactly 1 or 2 when it
can be inferred from the hard-call.  The exceptions are (i) when a fixed-width
encoding is used (bit 5 clear and bit 6 set in the variant record type) and
(ii) when multiallelic or phased dosage is stored for that sample.

As for auxiliary data track \#3:

\begin{itemize}
\item If bit 5 is set and bit 6 is clear in the variant record type, track \#3
  is a difflist indicating which samples have dosage information.
\item If bit 5 is clear and bit 6 is set, track \#4 has an entry for every
  single sample, and a value of 65535 in that array represents a missing
  dosage.  Track \#3 doesn't exist.
\item If bits 5 and 6 are both set, track \#3 is a bitarray with $N$ bits,
  where a set bit indicates the corresponding sample has a dosage in track \#4.
\end{itemize}

For example, suppose $N=488377$, sample ID 10000 has an ALT allele dosage of
1.5 on a 0..2 scale, every other even sample ID has an ALT allele dosage of
0.75, and no odd sample IDs have associated dosages.  Then a bitarray is a more
compact way to represent the sample ID list than a difflist, so bits 5 and 6 of
the variant record type would normally be set.  Auxiliary data track \#3 would
contain 61047 bytes with value \texttt{0x55}, followed by a byte with value
\texttt{0x01}.  Auxiliary data track \#4 would start with 5000 instances of
\texttt{0x00 0x30}, which is the little-endian \texttt{uint16} encoding of
$(0.75/2)\cdot 2^{15}$, followed by \texttt{0x00 0x60} representing the lone
1.5 dosage (since sample ID 10000 is the 5001th smallest ID with a dosage),
followed by 239188 more instances of \texttt{0x00 0x30}.

\subsubsection{Multiallelic dosages}

When dosage data is present and the variant has multiple ALT alleles, auxiliary
data tracks \#5-6 indicate how the total-ALT-allele-dosages stored in track \#4
are divided between the ALT alleles.

The planned format of these data tracks is described in
\texttt{pgenlib\_internal} comments.  This will not be considered finalized
until the \texttt{pgenlib\_internal} implementation is complete and has
undergone some testing.

\subsubsection{Phased dosages}

The PGEN format supports an \textit{implicit} dosage-phase representation.
Consider the scenario where a haplotype dosage is equal to exactly 0 or 1, the
corresponding hard-call is heterozygous, its phase is stored, and the variant
is biallelic.  Then, there is only one way to split the dosage into
left-haplotype and right-haplotype components that is consistent with the
phased hard-call and makes one of the components an integer.  For example, let
the left-haplotype dosage be 1 and the right-haplotype dosage be 0.01, and
check how it is represented in the previous data tracks: we'd already have a
1$|$0 hard-call and a total ALT dosage of 1.01.  The only ways to split a total
ALT dosage of 1.01 into left- and right-haplotype components where both values
are in [0, 1] and one of them is an integer are (left = 1, right = 0.01) and
(left = 0.01, right = 1), and only the first one is consistent with the phased
hard-call.  Therefore, when no explicit dosage-phase is stored for a biallelic
variant in this scenario, the dosage is considered implicitly phased in this
manner.  This can save a significant amount of disk space.

Of course, not all dosage-phase information is of this convenient form.  To
cover the other bases...

When bit 7 of the variant record type is set, auxiliary data track \#8 stores
explicit dosage phasing information, represented as an array of
\texttt{int16}s.  In this array, a value in [$-2^{14}$, $2^{14}$] is $2^{14}$
multiplied by ([left-haplotype ALT dosage] minus [right-haplotype ALT dosage]).

As for auxiliary data track \#7:

\begin{itemize}
\item If bit 5 is clear and bit 6 is set in the variant record type, track \#8
  has an entry for every single sample, and a value of -32768 in that array
  represents a missing dosage.  Track \#7 doesn't exist.
\item Otherwise, let $D$ be the number of entries in track \#4.  Auxiliary data
  track \#7 is a bitarray with $D$ bits, where a bit is set iff the
  corresponding dosage has a dosage-phase value stored in track \#8.
\end{itemize}

\subsubsection{Phased multiallelic dosages}

When phased dosage data is present and the variant has multiple ALT alleles,
auxiliary data tracks \#9-10 store the additional information needed to resolve
all per-haplotype allele dosages.  This part of the specification will not be
filled in until the \texttt{pgenlib\_internal} implementation is complete and
has undergone some testing.

\newpage
\section{PSAM Format Specification}

PSAM is a text format which stores sample information.  Its key properties are:

\begin{itemize}
\item A PLINK 1 .fam file is a valid PSAM file, excepting the pathological case
  where some FIDs or IIDs start with the prohibited '\#' character.
\item An optional ``source ID'' (SID for short) can be used to distinguish
  between samples derived from the same individual.
\item Zero, or many, phenotypes are now ok.
\item Oxford .sample files can be converted to PSAM without loss of
  information.  In particular, categorical covariates are now supported.
\end{itemize}

Fields are tab- or space-delimited; consecutive tabs/spaces are treated as a
single delimiter.  Newlines must be Unix- ('\texttt{\textbackslash n}') or
Windows-style ('\texttt{\textbackslash r\textbackslash n}').

\subsection{Header}

All initial lines starting with '\#', and no other lines, are part of the
header.

If no header lines are present, the file is treated as if it had a
``\texttt{\#FID IID PAT MAT SEX PHENO1}'' header line if it has 6 or more
columns, and ``\texttt{\#FID IID PAT MAT SEX}'' if there are exactly five.
This provides compatibility with (a slight generalization of) PLINK 1 .fam.

Otherwise:

\begin{itemize}
\item The last header line must start with ``\#FID'' or ``\#IID'', and no
  earlier header lines can start this way.  This header line labels the columns
  in the remainder of the file.
\item In this version of the specification, the earlier header lines don't
  matter.  Without loss of generality, the rest of this specification assumes
  there is only one header line.
\item If the FID column appears at all, it must be first.  The IID column is
  always present, and must be either first or immediately follow the FID
  column.
\item The other predefined columns are ``SID'', ``PAT'', ``MAT'', and ``SEX'';
  these are all optional, but when they do appear they are interpreted as
  described in the Body subsection.  Any other value in the header line is
  treated as a phenotype/covariate name.
\item Duplicate column headers are not allowed.
\end{itemize}

\subsection{Body}

Every nonempty line in the body must have at least as many fields as the
(possibly implicit) header line.  However, it's okay for these later lines to
have \textit{more} fields; the extras are ignored.  (This allows PLINK 1 .ped
files to be treated as PSAMs.)

Sample IDs are now tripartite (family ID-individual ID-source ID); if no FID
column is present, all FIDs are considered to be '0', and the analogous thing
is true when no SID column is present.  IID values are not permitted to be '0'.
Each full sample ID must be unique in the file, but it's fine for two of the
three components to match.

The PAT and MAT columns must either both be present, or both be absent.  When
they're present, they indicate the IIDs of the sample's parents (identical FID
is assumed); unknown parents are represented by '0' (this is why IID values
can't be '0').  It's fine if no sample ID corresponds to a parent mentioned in
the PAT or MAT column.

The SEX column encodes sex information.  '1', 'M', and 'm' are valid
representations of male sex.  '2', 'F', and 'f' are valid representations of
female sex.  Other values in this column should be interpreted as missing.
``NA'' is the preferred representation of a missing value in this column.

Phenotypes can be binary, quantitative, or categorical.  In general, which
class a particular phenotype is in must be inferred from the values in the
column (this is unfortunately necessary for backward compatibility):

\begin{itemize}
\item Values in a categorical-phenotype column are not permitted to start with
  a digit (or a digit preceded by a decimal point and/or '+'/'-'), or be equal
  to any capitalization of ``NA'' or ``nan''.  ``NONE'' should be used to
  represent a missing categorical-phenotype value.  This way, categorical
  phenotypes can always be distinguished from non-categorical phenotypes after
  seeing a single entry in the column.
\item The usual PLINK 1 encoding of binary phenotypes is control = 1, case = 2,
  and missing-value = -9/0/``NA''/``nan'' (among these, ``NA'' is now the
  preferred representation of missing values in PSAM files).  Therefore, PLINK
  2 defaults to interpreting any phenotype column containing only these values
  as binary, and any phenotype column containing a numeric value outside \{-9,
  0, 1, 2\} as quantitative.
\end{itemize}

\newpage
\section{PVAR Format Specification}

PVAR is a text format which stores non-genotype variant information.  Its key
properties are:

\begin{itemize}
\item A PLINK 1 .bim file is a valid PVAR file.
\item A VCF file is almost always a valid PVAR file.
\end{itemize}

Fields are tab- or space-delimited; consecutive tabs/spaces are treated as a
single delimiter.  Newlines must be Unix- ('\texttt{\textbackslash n}') or
Windows-style ('\texttt{\textbackslash r\textbackslash n}').

\subsection{Header}

All initial lines starting with '\#', and no other lines, are part of the
header.

If no header lines are present, the file is treated as if it had a
``\texttt{\#CHROM ID CM POS ALT REF}'' header line if it has 6 or more columns,
and ``\texttt{\#CHROM ID POS ALT REF}'' if there are exactly five.  This
provides compatibility with (a slight generalization of) PLINK 1 .bim.

Otherwise:

\begin{itemize}
\item The last header line must start with ``\#CHROM'', and no earlier header
  lines can start with that.  This header line labels the columns in the
  remainder of the file.
\item Only the following other column labels are recognized: ``POS'', ``ID'',
  ``REF'', ``ALT'', ``QUAL'', ``FILTER'', ``INFO'', ``CM'', and ``FORMAT''.  To
  allow regular VCF files to be treated as if they were PVAR files, ``FORMAT''
  causes parsing of the last header line to end (the FORMAT column is also
  ignored).  Other column labels are not permitted before ``FORMAT''.
\item If the FILTER or INFO columns are nonempty, the header should contain
  meta-information lines describing all FILTERs and INFO fields, following the
  VCF 4.3 specification.
\end{itemize}

\subsection{Body}

``CM'' indicates a column of centimorgan positions.  All other columns are as
defined in the VCF specification, with the additional restriction that the
space character is not permitted in the INFO field.

\end{document}
